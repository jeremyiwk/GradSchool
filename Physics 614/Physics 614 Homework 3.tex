\documentclass[]{article}
%\usepackage{setspace}
%\onehalfspacing
\usepackage{amsmath,amssymb,amsthm}
\renewcommand{\qedsymbol}{$\blacksquare$}
\usepackage{amsmath}
\usepackage{bm}
\usepackage{amsfonts}
\usepackage{mathrsfs}
\usepackage{amssymb}
\usepackage{enumerate}
\usepackage{mdwlist}
\usepackage{dirtytalk}
\usepackage{xparse}
\usepackage{physics}
\usepackage{graphicx}
\setcounter{MaxMatrixCols}{13}
\setlength\parindent{0pt}
\usepackage[none]{hyphenat}
\usepackage[hmarginratio=1:1]{geometry}
\begin{document}

{\Large Physics 614 Homework 3}\\
{Jeremy Welsh-Kavan}\\
%\end{center}
\vspace{0.2 cm}
\hfill \\
\noindent\rule{15cm}{0.4pt} \\

\begin{enumerate}[1.]

\item {\bf  Bose-Einstein condensation in 2D}

Recall that the average number of particles in a Bose gas is given by

\begin{equation}
\begin{split}
N & = \sum_r \frac{1}{ e^{\beta(E_r-\mu)} - 1   }
\end{split}
\end{equation}

which we can rewrite for the 2D case using the density of states, $g(\epsilon) = \frac{L^2m}{2\pi \hbar^2}$, and rearrange to find the density. With $z = e^{\beta\mu}$, the fugacity, we have

\begin{equation}
\begin{split}
N & =  \int_{0}^{\infty} d\epsilon \frac{ g(\epsilon) }{ z^{-1} e^{\beta\epsilon} -1  } \\
N & =  \frac{L^2m}{2\pi \hbar^2}\int_{0}^{\infty} d\epsilon \frac{ 1}{ z^{-1} e^{\beta\epsilon} -1  } \\
N & =  \frac{L^2mk_BT}{2\pi \hbar^2}\int_{0}^{\infty} dy \frac{ 1}{ z^{-1} e^{y} -1  } \\
\rho & =  \frac{mk_BT}{2\pi \hbar^2} g_1(z)   \\
\rho & =  \frac{1}{ \lambda^2} g_1(z)   \\
\end{split}
\end{equation}

where $g_\alpha(z) = \frac{1}{(\alpha-1)!} \int_{0}^{\infty} dy\frac{y^{\alpha-1}}{ z^{-1}e^y -1}$ are the Bose functions. Suppose there exists $T_c$ such that the continuum approximation in (2) breaks down. That is, suppose there is $T_c$ such that

\begin{equation}
\begin{split}
\rho \frac{2\pi \hbar^2}{mk_BT_c} & =  g_1(1)   \\
\end{split}
\end{equation}

Then we would have

\begin{equation}
\begin{split}
\rho \frac{2\pi \hbar^2}{mk_BT_c} & = \zeta(1) = \sum_{n=1}^{\infty} \frac{1}{n} =\infty  \\
\end{split}
\end{equation}

but then $T_c$ must be zero. So there is no Bose-Einstein condensation transition in
two dimensions.

\hfill \\
\noindent\rule{15cm}{0.4pt} \\

\item {\bf Two-state Bose-Einstein condensation}
\hfill \\
Consider an ideal Bose gas consisting of molecules with two internal states, the ground state with energy
$\epsilon_0 = 0$ and an excited state with energy $\epsilon_1 > 0$. 

\begin{enumerate}[i.]

\item We can write the grand canonical partition function as

\begin{equation}
\begin{split}
\mathcal{Q} & = \prod_\alpha \mathcal{Z}_\alpha \\
\mathcal{Q} & = \prod_\alpha \frac{  1 }{ 1-  e^{-\beta(\epsilon_\alpha -\mu )}  } \\
\end{split}
\end{equation}

where $\alpha$ indexes the many-particle eigenstates of the Hamiltonian, $\mathcal{H}$. We can divide these states into two categories: one with energy $\epsilon_{\bm{k}}$, for each allowed $\bm{k}$ value, and the other with energy $\epsilon_{\bm{k}} + \epsilon_1$, in which the internal degree of freedom is excited. In this case the grand partition function becomes

\begin{equation}
\begin{split}
\mathcal{Q} & = \prod_{\bm{k}} \frac{  1 }{ \left[ 1-  e^{-\beta(\epsilon_{\bm{k}} -\mu )}\right] \left[  1-  e^{-\beta(\epsilon_{\bm{k}} + \epsilon_1 -\mu )}  \right]  }\\
\end{split}
\end{equation}

\item Recall that the average number of particles $N$ can be written in terms of $\mathcal{Q}$ as $N = \frac{1}{\beta} \pdv{\ln\mathcal{Q}}{\mu}$. Using the expression for $\mathcal{Q}$ found in (4), and setting $z_1 = e^{\beta\mu}$ and $z_2 = e^{\beta(\mu - \epsilon_1)}$, we have

\begin{equation}
\begin{split}
N & = \frac{1}{\beta} \pdv{\ln\mathcal{Q}}{\mu} \\
N & = - \frac{1}{\beta} \pdv{\mu} \left( \sum_{\bm{k}} \ln\left(   1-  e^{-\beta(\epsilon_{\bm{k}} -\mu )} \right) + \ln\left(   1-  e^{-\beta(\epsilon_{\bm{k}} + \epsilon_1 -\mu )}   \right) \right) \\
N & =  \frac{1}{\beta} \left( \sum_{\bm{k}} \frac{  \beta e^{-\beta(\epsilon_{\bm{k}} -\mu )}  }{  1-  e^{-\beta(\epsilon_{\bm{k}} -\mu )} } + \frac{ \beta e^{-\beta(\epsilon_{\bm{k}} + \epsilon_1 -\mu )}  }{   1-  e^{-\beta(\epsilon_{\bm{k}} + \epsilon_1 -\mu )}   } \right) \\
N & = \sum_{\bm{k}} \frac{  z_1 e^{-\beta\epsilon_{\bm{k}}}  }{  1- z_1 e^{-\beta\epsilon_{\bm{k}}} } +  \sum_{\bm{k}}  \frac{ z_2 e^{-\beta\epsilon_{\bm{k}} }  }{   1-  z_2 e^{-\beta\epsilon_{\bm{k}} }   }  \\
N & = \sum_{\bm{k}} \frac{ 1 }{   z_1^{-1} e^{\beta\epsilon_{\bm{k}}}   - 1} +  \sum_{\bm{k}}  \frac{ 1 }{   z_2^{-1} e^{\beta\epsilon_{\bm{k}} }  -  1  }  \\
\end{split}
\end{equation}

Now, using the density of states $g(\epsilon) =  \frac{V}{4\pi^2} \left( \frac{2m}{\hbar^2} \right)^{3/2}\epsilon^{1/2}  $, we can rewrite the sums in (5) as integrals over $\epsilon$

\begin{equation}
\begin{split}
N & = \sum_{\bm{k}} \frac{ 1 }{   z_1^{-1} e^{\beta\epsilon_{\bm{k}}}   - 1} +  \sum_{\bm{k}}  \frac{ 1 }{   z_2^{-1} e^{\beta\epsilon_{\bm{k}} }  -  1  }  \\
N & \approx \frac{V}{4\pi^2} \left( \frac{2m}{\hbar^2} \right)^{3/2}  \left[  \int d\epsilon \frac{ \epsilon^{1/2} }{  z_1^{-1} e^{\beta\epsilon}  -  1  } + \int d\epsilon \frac{ \epsilon^{1/2} }{  z_2^{-1} e^{\beta\epsilon}  -  1  }  \right] \\
N & = \frac{V}{4\pi^2} \left( \frac{2m}{\beta\hbar^2} \right)^{3/2} \left( \frac{1}{2}  \right)! \left[ \frac{1}{(1/2)!} \int dy  \frac{ y^{1/2} }{  z_1^{-1} e^{y}  -  1  } + \frac{1}{(1/2)!}\int dy \frac{ y^{1/2} }{  z_2^{-1} e^{y}  -  1  }  \right] \\
N & = \frac{V}{4\pi^2} \left( \frac{2m}{\beta\hbar^2} \right)^{3/2}  \frac{\sqrt{\pi}}{2}  \left( g_{3/2}(z_1) + g_{3/2}(z_2)  \right) \\
\rho & =  \left( \frac{mk_BT}{2\pi\hbar^2} \right)^{3/2}   \left( g_{3/2}(z_1) + g_{3/2}(z_2)  \right) \\
\rho & = \frac{1}{\lambda^3}  \left( g_{3/2}(z_1) + g_{3/2}(z_2)  \right) \\
\rho\lambda^3 & =   g_{3/2}(z_1) + g_{3/2}(z_2) \\
\end{split}
\end{equation}

\item In the low temperature limit $z_1 \to 1$ while $\lambda$ increases without bound. So there must be some $T_c$ at which the approximation in (5) breaks down. Setting $z_1 = 1$, we have

\begin{equation}
\begin{split}
\rho  \left( \frac{2\pi\hbar^2} {mk_BT_c}\right)^{3/2}  & =   g_{3/2}(z_1) + g_{3/2}(z_2) \\
\rho \left( \frac{2\pi\hbar^2} {mk_BT_c}\right)^{3/2}  & =   g_{3/2}(z_1) + g_{3/2}(z_1 e^{-\beta\epsilon_1}) \\
\rho \left( \frac{2\pi\hbar^2}{mk_BT_c} \right)^{3/2}  & =   g_{3/2}(1) + g_{3/2}( e^{-\epsilon_1/(k_BT_c)}) \\
\end{split}
\end{equation}

The solution to equation (7) yields the critical temperature, $T_c$, for condensation in this system. \\
Let $T_c^0$ be the critical temperature for the same gas of molecules with all molecules restricted to the ground state (i.e. the solution to $(h/\sqrt{2\pi m k_BT_c^0})^3\rho = \zeta(3/2)$). 

\begin{itemize}

\item In the low temperature limit, if $T_c^0$ solves $(h/\sqrt{2\pi m k_BT_c^0})^3\rho = \zeta(3/2)$ then $T_c^0$ solves (7) to a good approximation. Therefore, when $k_BT_c^0 \ll \epsilon_1$, we have 

\begin{equation}
\begin{split}
\left( \frac{T_c}{T_c^0} \right)^{3/2} & = \frac{  \zeta(3/2) }{ \zeta(3/2) +  g_{3/2}( e^{-\epsilon_1/(k_BT_c^0)})  } \\
\frac{T_c}{T_c^0} & \approx \frac{ 1 }{ \left( 1+   \frac{1}{\zeta(3/2) }e^{-\epsilon_1/(k_BT_c^0)} \right)^{2/3} } \\
\frac{T_c}{T_c^0} & \approx 1 - \frac{2}{3\zeta(3/2) }e^{-\epsilon_1/(k_BT_c^0)} \\
\end{split}
\end{equation}

\item When $k_BT_c^0 \gg \epsilon_1$, 

\begin{equation}
\begin{split}
\frac{T_c}{T_c^0} & = \left( \frac{  \zeta(3/2) }{ \zeta(3/2) +  g_{3/2}( e^{-\epsilon_1/(k_BT_c^0)})  } \right)^{2/3} \\
\frac{T_c}{T_c^0} & = \left( \frac{ 1}{ 1 +  \frac{1}{ \zeta(3/2)} ( \Gamma(-1/2)(\epsilon_1/(k_BT_c^0)^{1/2} + \zeta(3/2) + \mathcal{O}(\epsilon_1/(k_BT_c^0)) )} \right)^{2/3} \\
\frac{T_c}{T_c^0} & = \left( \frac{ 1}{ 2 +  \frac{1}{ \zeta(3/2)} (  -2(\pi\epsilon_1/(k_BT_c^0))^{1/2} + \zeta(3/2) )} \right)^{2/3} \\
\frac{T_c}{T_c^0} & = 2^{-2/3}\left( \frac{ 1}{ 1 -  \frac{1}{ \zeta(3/2)} \sqrt{ \pi\epsilon_1/(k_BT_c^0)}} \right)^{2/3} \\
\frac{T_c}{T_c^0} & \approx 2^{-2/3}\left(  1 + \frac{2}{ 3\zeta(3/2)} \sqrt{ \pi\epsilon_1/(k_BT_c^0)}   \right)\\
\end{split}
\end{equation}

where we have omitted a factor of $2^{4/3}$ due primarily to confusion. 


\end{itemize}




%\rho \left( \frac{2\pi\hbar^2}{mk_BT_c} \right)^{3/2}  & \approx   \zeta(3/2) + e^{-\epsilon_1/k_BT_c}

\end{enumerate}

\hfill \\
\noindent\rule{15cm}{0.4pt} \\

\item {\bf Density of states for 1d phonons}

Consider a linear chain of $N$ point masses $m$ confined to move in one dimension, and connected to their nearest neighbors with harmonic bonds of spring constant $\kappa$ and rest length $a$. The standard harmonic analysis of classical mechanics shows that a complete basis to describe the displacements of the masses at positions $x = ia$, $i \in (0,1, \dots, N-1)$ is provided by plane-wave-like normal modes

\begin{equation}
\begin{split}
u(x) & = \frac{1}{\sqrt{L}} e^{ikx}
\end{split}
\end{equation}

where $k = \frac{2\pi}{L}n$, $n = 0, \pm1, \pm2, \dots , \pm (N-1)/2$. In contrast to the continuous medium, the lattice constant imposes a finite range to the momenta, $-\pi/a<k<\pi/a$ in the limit of large $N$. Classically, each mode has a momentum-dependent oscillation frequency, $\omega_k = \omega_0 | \sin(ka/2) |$, where $\omega_0 = 2\sqrt{\kappa/m}$.


\begin{enumerate}[i.]

\item For a sum over $k$ of a function $F(\omega_k)$, we have $\sum_k F(\omega_k) \approx \frac{L}{2\pi} \int_{-\pi/a}^{\pi/a} dk F(\omega_k) = \frac{L}{\pi} \int_{0}^{\pi/a} dk F(\omega_k) $ since $\omega_k$ is even on the interval $-\pi/a< k < \pi/a$. We can rewrite the differential $dk$ as follows 

\begin{equation}
\begin{split}
\omega_k & = \omega_0 \left| \sin(\frac{ka}{2}) \right| \\
\omega_k & = \omega_0 \sin(\frac{ka}{2})  \; \text{, for } \;  k < \pi/a \\
\dv{\omega_k }{k} & =  \frac{\omega_0a}{2} \cos(\frac{ka}{2}) \\
d\omega_k & = \frac{\omega_0a}{2}  \sqrt{1  -  \left( \frac{\omega_k }{\omega_0} \right)^2} dk  \\
dk & = \frac{2}{\omega_0a} \frac{d\omega_k }{ \sqrt{1  -  \left( \frac{\omega_k }{\omega_0} \right)^2}  }
\end{split}
\end{equation}

With the prefactors from the integration, we have

\begin{equation}
\begin{split}
g(\omega) & = \frac{2L}{\pi a \omega_0} \frac{1 }{ \sqrt{1  -  \left( \frac{\omega }{\omega_0} \right)^2}  }
\end{split}
\end{equation}

\item 

\begin{frame}{}
    \begin{figure}[h]
        \begin{minipage}[b]{0.5\linewidth}
            \centering
            \includegraphics[width=\textwidth]{1D_osc_Dos.pdf}
        \end{minipage}
        \caption{Plot of the density of states as a function of $\omega/\omega_0$. Sorry this plot is so small. }
    \end{figure}
\end{frame}

The singularity in this plot is due to the fact that, at large $N$, we have $\omega/\omega_0 = 1$. 


\item As $k\to 0$, $ \omega_k \to vk$, where $v = \omega_0 a/2$. In this limit, $\frac{1}{v}d\omega_k =  dk$. So $g(\omega) = \frac{L}{v\pi}$.

\end{enumerate}

\end{enumerate}



\begin{center}
\noindent\rule{15cm}{0.4pt} \\
\end{center}
$$\clubsuit$$
\end{document}










%\begin{equation}
%\begin{split}
%\end{split}
%\end{equation}
