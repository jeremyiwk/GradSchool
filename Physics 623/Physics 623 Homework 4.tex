\documentclass[]{article}
%\usepackage{setspace}
%\onehalfspacing
\usepackage{amsmath,amssymb,amsthm}
\renewcommand{\qedsymbol}{$\blacksquare$}
\usepackage{amsmath}
\usepackage{bm}
\usepackage{amsfonts}
\usepackage{mathrsfs}
\usepackage{amssymb}
\usepackage{enumerate}
\usepackage{mdwlist}
\usepackage{dirtytalk}
\usepackage{xparse}
\usepackage{physics}
\usepackage{xfrac}
\usepackage{graphicx}
\setcounter{MaxMatrixCols}{13}
\setlength\parindent{0pt}
\usepackage[none]{hyphenat}
\usepackage[hmarginratio=1:1]{geometry}
\begin{document}

{\Large Physics 623 Homework 4}\\
{Jeremy Welsh-Kavan}\\
%\end{center}
%\vspace{0.2 cm}
\hfill \\ 
\noindent\rule{15cm}{0.4pt} \\

4.3.1. {\bf Potentials in Coulomb gauge } \\




\hfill \\ 
\noindent\rule{15cm}{0.4pt} \\

4.3.2. {\bf Radiation from cyclotron motion } \\

Consider a point mass $m$ with charge $e$ that moves in a plane perpendicular to a homogeneous magnetic field $B$ and assume the particle travels with a nonrelativistic velocity, $v \ll c$. \\

\begin{enumerate}[a)]


\item Since the particle is confined to the plane we let $\bm{r}(t) = (x(t), y(t), 0)$ and $\bm{B} = (0,0,B)$. We can use the Lorentz force law to find the velocity, $\bm{v}(t)$. 

\begin{equation}
\begin{split}
\dot{\bm{v}} & = \frac{e}{m} \bm{v} \times \bm{B} \\
\implies \dot{\bm{v}}^2 & = \frac{e^2 v^2 B^2}{m^2}  \\
\end{split}
\end{equation}

Then, using the Larmor formula, we have that the power, $\mathscr{P}$, radiated by the particle is

\begin{equation}
\begin{split}
\mathscr{P} & = \frac{2e^2}{3c^3}  \dot{\bm{v}}^2 \\
\mathscr{P} & = \frac{2 e^4 v^2\bm{B}^2}{3 c^3m^2}  \\
\mathscr{P} & = \frac{4 e^4 B^2}{3 c^3m^3} E \\
\end{split}
\end{equation}

where $E$ is the kinetic energy of the particle. \\

\item Since (2) represents the power radiated by the particle, the power lost by the particle is the negative time derivative of the kinetic energy. We can write (2) as a first order differential equation in $E(t)$. This yields

\begin{equation}
\begin{split}
\dv{E(t)}{t} & =  -   \frac{4 e^4 B^2}{3 c^3m^3} E(t) \\
\implies E(t) & =  E_0 e^{ -  t/\tau}
\end{split}
\end{equation}

where $\tau = \frac{3 c^3m^3}{4 e^4 B^2}$ and $E_0$ is the initial kinetic energy of the charge. \\

\item Let $e$ and $m$ be the charge and mass of an electron respectively and let $B = 1$ Tesla. Then the timescale of energy loss, $\tau$, is

\begin{equation}
\begin{split}
\tau & = \frac{3 c^3m^3}{4 e^4 B^2} \\
\tau & =  4\pi\epsilon_0 \frac{3 c^3m^3}{4 e^4 B^2} \\
\tau & \approx 2.6 \; \text{s} \\ 
\end{split}
\end{equation}

\end{enumerate}

\hfill \\ 
\noindent\rule{15cm}{0.4pt} \\

4.3.3. {\bf Radiating harmonic oscilator } \\

Consider particle with charge $e$ and mass $m$ in a one-dimensional harmonic potential. Let the frequency of the harmonic oscillator by $\omega_0$.

\begin{enumerate}[a)]

\item We can again rewrite the Larmor formula to express the power, $\mathscr{P}$, in terms of the total energy of the particle, $E$. First, we rewrite $ \dot{\bm{v}}^2$ in terms of the potential energy.s


\begin{equation}
\begin{split}
\dot{\bm{v}} & = - \omega_0^2 \bm{x} \\
\dot{\bm{v}}^2 & = \omega_0^4\bm{x}^2 \\
\dot{\bm{v}}^2 & =  \frac{2  \omega_0^2 }{m} V(x) \\
\implies \overline{\dot{\bm{v}}^2} & =  \frac{2  \omega_0^2 }{m} \overline{V} \\
\end{split}
\end{equation}

Using the fact that $\overline{V} = \overline{T} = E/2$, where $E$ is the total energy of the particle, we have that the power radiated by the particle averaged over on oscillation period is


\begin{equation}
\begin{split}
\mathscr{P} & = \frac{2e^2}{3c^3} \overline{\dot{\bm{v}}^2} \\
\mathscr{P} & = \frac{4\omega_0^2e^2}{3mc^3} \overline{V}\\ 
\mathscr{P} & = \frac{2\omega_0^2e^2}{3mc^3} E \\ 
\end{split}
\end{equation}

\item The rate of energy loss is again $-\mathscr{P}$ so we have


\begin{equation}
\begin{split}
\dv{E}{t} & = - \frac{2\omega_0^2e^2}{3mc^3} E \\ 
\implies E(t) & = E_0 e^{-t/\tau} \\
\end{split}
\end{equation}

where $\tau = \frac{3mc^3}{2\omega_0^2e^2}$ and $E_0$ is the initial energy of the particle. \\

\item With $e$ and $m$ the charge and mass of the electron and $\omega_0 = 10^{15} \; \text{sec}^{-1}$, the timescale, $\tau$ is

\begin{equation}
\begin{split}
\tau & =  \frac{3mc^3}{2\omega_0^2e^2} \\
\tau & \approx 1.6\times10^{-7} \; \text{s} \\
\end{split}
\end{equation}



\end{enumerate}




\hfill \\ 
\noindent\rule{15cm}{0.4pt} \\

$$\clubsuit$$
\end{document}





%+ \lambda \frac{\hbar c\alpha}{|\bm{r}_1 - \bm{r}_2|}




%\begin{equation}
%\begin{split}
%\end{split}
%\end{equation}
