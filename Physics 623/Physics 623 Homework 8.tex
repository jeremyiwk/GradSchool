\documentclass[]{article}
%\usepackage{setspace}
%\onehalfspacing
\usepackage{amsmath,amssymb,amsthm}
\renewcommand{\qedsymbol}{$\blacksquare$}
\usepackage{amsmath}
\usepackage{amsfonts}
\usepackage{mathrsfs}
\usepackage{amssymb}
\usepackage{bm}
\usepackage{enumerate}
\usepackage{mdwlist}
\usepackage{dirtytalk}
\usepackage{xparse}
\usepackage{physics}
\usepackage[cmtip,all]{xy}
\newcommand{\longsquiggly}{\xymatrix{{}\ar@{~>}[r]&{}}}
\usepackage{graphicx}
\usepackage{xcolor}% http://ctan.org/pkg/xcolor
\usepackage{hyperref}% http://ctan.org/pkg/hyperref
\hypersetup{
  colorlinks=true,
  linkcolor=blue!50!red,
  urlcolor=green!70!black
}
\setcounter{MaxMatrixCols}{13}
\setlength\parindent{0pt}
\usepackage[none]{hyphenat}
\usepackage[hmarginratio=1:1]{geometry}
\begin{document}

{\Large Physics 623 Homework 8}\\
{Jeremy Welsh-Kavan}\\
\hfill \\
\noindent\rule{15cm}{0.4pt} \\

4.6.2. {\bf Synchrotron radiation} (continued) \\



c)  We know from problem 4.6.1 that the Bessel function $J_{2m}(x)$ can be written \\

\begin{equation}
\begin{aligned}
J_{2m} (x)  & = \frac{1}{\pi} \int_{0}^{\pi} d\phi \: \cos( x \sin( \phi/2)) \cos(m \phi) \\
\end{aligned}
\end{equation} \\

By part b), $P_m$ can be expressed as \\

\begin{equation}
\begin{aligned}
P_m & = \frac{ e^2 \omega_0 }{ R} \frac{ m }{  \pi } \int_{0}^{\pi} d\phi \: \cos( m\phi )\left[ \beta^2 - 1 - 2 \beta^2 \sin^2( \phi/2 )  \right]   \frac{ \sin( 2m \beta \sin( \phi/2) )  }{ \sin( \phi/2)   } \\
\end{aligned} 
\end{equation} \\

where $\omega_0 R /c = \beta$. We now rewrite this as a sum of integrals. Define $A : = e^2 \omega_0 m /R$. Then we have \\

\begin{equation}
\begin{aligned}
P_m & = \frac{ A }{  \pi } \int_{0}^{\pi} d\phi \: \cos( m\phi ) \left[ \beta^2 - 1 - 2 \beta^2 \sin^2( \phi/2)  \right]   \frac{ \sin( 2m \beta \sin( \phi/2) )  }{ \sin( \phi/2 )  } \\
\frac{ P_m }{ A} & =  (\beta^2 - 1 ) \frac{ 1 }{  \pi } \int_{0}^{\pi} d\phi \: \cos( m\phi ) \frac{ \sin( 2m \beta \sin( \phi/2) )  }{ \sin( \phi/2 )  }  - 2 \beta^2   \frac{ 1 }{  \pi } \int_{0}^{\pi} d\phi \: \sin^2( \phi/2)     \frac{ \sin( 2m \beta \sin( \phi/2) )  }{ \sin( \phi/2 )  } \\
\frac{ P_m }{ A} & =  (\beta^2 - 1 ) \frac{ 1 }{  \pi } \int_{0}^{\pi} d\phi \: \cos( m\phi ) \frac{ \sin( 2m \beta \sin( \phi/2) )  }{ \sin( \phi/2 )  }  - 2 \beta^2   \frac{ 1 }{  \pi } \int_{0}^{\pi} d\phi \: \cos( m\phi )  \sin( \phi/2)   \sin( 2m \beta \sin( \phi/2) ) \\
\end{aligned} 
\end{equation} \\

Now with $\xi = 2m \beta$, we can write these in terms of the Bessel functions as \\

\begin{equation}
\begin{aligned}
\frac{ P_m }{ A} & =  (\beta^2 - 1 ) \frac{ 1 }{  \pi } \int_{0}^{\pi} d\phi \:  \frac{ \partial^{-1} }{ \partial \xi^{-1} }\cos( m\phi )  \cos( \xi \sin( \phi/2) )   +  2 \beta^2   \frac{ 1 }{  \pi } \int_{0}^{\pi} d\phi \: \frac{ \partial }{ \partial \xi } \cos( m\phi )   \cos( \xi \sin( \phi/2) ) \\
& = (\beta^2 - 1 )  \frac{ \partial^{-1} }{ \partial \xi^{-1} } J_{2m}(\xi)  +  2 \beta^2   \frac{ \partial }{ \partial \xi } J_{2m}(\xi) \\
\end{aligned} 
\end{equation} \\

Replacing the constants, we have \\

\begin{equation}
\begin{aligned}
P_m & = \frac{ e^2 \omega_0 m  }{ R}  \left[  2 \beta^2    J'_{2m}( 2 m \beta ) -  ( 1 - \beta^2  ) \int_{0}^{2m\beta} dx \:  J_{2m}(x)  \right]    \\
\end{aligned} 
\end{equation} \\

as desired. \\

\hfill \\

d) For $\beta \approx 1$, the second term vanishes and the primary contribution comes from the first term. Therefore, for $\beta \approx 1$ we have

\begin{equation}
\begin{aligned}
P_m & \approx \frac{ e^2 \omega_0 m  }{ R}  2 \beta^2    J'_{2m}( 2 m \beta )    \\
\end{aligned} 
\end{equation} \\

By the asymptotic form of $ J'_{2m}( 2 m \beta )$ derived in the last homework, we have for $2m\gg 1$ and $\beta \approx 1$,  \\

\begin{equation}
\begin{aligned}
P_m & \approx \frac{ e^2 \omega_0 \beta^2   }{ R}  2 m   
\begin{cases}
\frac{ 2^{2/3} }{ 3^{1/3} \Gamma(1/3) } (2m)^{-2/3} \; \; \; \text{for } \; 1 \ll m \ll \gamma^3 \\
\frac{ 1}{ \sqrt{ 2\pi }} \sqrt{ \gamma / 2m} e^{ - 2m / 3\gamma^3 } \; \; \; \text{for } \;  m \gg \gamma^3 \\
\end{cases}   \\
%
P_m & \approx \frac{ e^2 \omega_0 \beta^2   }{ R}   
\begin{cases}
\frac{ 2 }{ 3^{1/3} \Gamma(1/3) } m^{1/3} \; \; \; \text{for } \; 1 \ll m \ll \gamma^3 \\
\frac{ 1}{ \sqrt{ 2\pi }} \sqrt{ 2m \gamma } e^{ - 2m / 3\gamma^3 } \; \; \; \text{for } \;  m \gg \gamma^3 \\
\end{cases}   \\
%
\end{aligned} 
\end{equation} \\

Since $P_m$ is an increasing function of $m$ in the first case and decreasing in the second case, $P_m$ must reach a maximum, $m_\text{max}$, when $m$ is of order $\gamma^3$. Therefore, $m_\text{max} \propto \gamma^3$.  \\

e) For the Advanced Light Source, electrons with $T = 1.9$ GeV of energy travel in a circle of radius roughly $R = 20$ m. From the previous problem, the power spectrum is peaked around $\omega_0 m$ where $m$ is of order $\gamma^3$. The energy yields a velocity, and hence an angular frequency, given by \\


\begin{equation}
\begin{aligned}
T & = \gamma m c^2  = \frac{ 1 }{ \sqrt{1 - v^2/c^2} } mc^2\\
v &= c \sqrt{ 1 - \frac{ m^2 c^4 }{ T^2 }} \\  
\omega_0 & = \frac{ v }{ R} \\
\end{aligned} 
\end{equation} \\

Therefore, the peak frequency for the Advanced Light Source is roughly \\

\begin{equation}
\begin{aligned}
\omega & = m \omega_0 \\
& \approx \gamma^3 \omega_0 \\
& = \gamma^3 \frac{c}{R} \sqrt{ 1 - \frac{ m^2 c^4 }{ T^2 }} \\
& = \frac{ c }{R} \sqrt{  \frac{ T^2 }{ m^2 c^4 } - 1 }\\
& \approx 55.7 \text{rad /s} \\
& \approx 8.87 \text{GHz} \\
\end{aligned} 
\end{equation} \\

with associated wavelength $\lambda \approx  0.0338 $ m... which I think should be in the X-ray range but is not. \\

















\newpage



\noindent\rule{15cm}{0.4pt} \\

$$\clubsuit$$

\end{document}











%\begin{equation}
%\begin{aligned}
%\end{aligned}
%\end{equation}
