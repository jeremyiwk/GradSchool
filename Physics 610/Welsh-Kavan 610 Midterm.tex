\documentclass[]{book}
\usepackage{amsmath,amssymb,amsthm}
\renewcommand{\qedsymbol}{$\blacksquare$}
\usepackage{amsmath}
\usepackage{amsfonts}
\usepackage{amssymb}
\usepackage{enumerate}
\usepackage{dirtytalk}
\setcounter{MaxMatrixCols}{13}
\setlength\parindent{0pt}
\usepackage[none]{hyphenat}
\usepackage[hmarginratio=1:1]{geometry}
\begin{document}
\begin{center}
{\Large Physics 610 Midterm Fall 2020}\\
{Jeremy Welsh-Kavan}\\
\end{center}
\vspace{0.2 cm}
\begin{center}
\noindent\rule{15cm}{0.4pt} \\
\end{center}
Let $X$ and $Y$ be sets.\\

{\bf Lemma 1:} Let $f:X\to Y$ and $g:Y\to X$ be mappings such that $f \circ g$ $=$ id$_Y$. Then $f$ is surjective and $g$ is injective.
\begin{proof}[Proof of Lemma 1]
Let $y \in Y$. Since $g$ is a mapping, there is $x \in X$ such that $g(y)=x$ and, since $f$ is a mapping, $f(x) \in Y$. But $f(x) = f(g(y)) = (f\circ g)(y)$ and, by our assumption, $f \circ g$ $=$ id$_Y$. So $(f\circ g)(y)=y$. Therefore, for each $y\in Y$ there exists $x\in X$ such that $f(x)=y$, so $f$ is surjective.
Let $y_1,y_2\in Y$ and assume $g(y_1)=g(y_2)$. Then $f(g(y_1))=f(g(y_2))$. But, by assumption $f \circ g$ $=$ id$_Y$, so $y_1=f(g(y_1))=f(g(y_2))=y_2$. Therefore, if $y_1,y_2\in Y$ and $g(y_1)=g(y_2)$ then $y_1=y_2$, so $g$ is injective.
\end{proof}

{\bf Lemma 2:} Let $f:X\to Y$ be surjective. Then there exists an injective mapping $g:Y\to X$ such that $f \circ g$ $=$ id$_Y$.
\begin{proof}[Proof of Lemma 2]
Define a relation $\sim$ on $X$ such that, for $x_1,x_2\in X$, $x_1\sim x_2$ if $f(x_1)=f(x_2)$. We can check that this is an equivalence relation. If $x_1,x_2,x_3\in X$ then $f(x_1)=f(x_1)$, and $f(x_1)=f(x_2)$ implies $f(x_2)=f(x_1)$, so $\sim$ is reflexive and symmetric. If $f(x_1)=f(x_2)$ and $f(x_2)=f(x_3)$ then $f(x_1)=f(x_2)=f(x_3)$ so $f(x_1)=f(x_3)$. Hence, $\sim$ is reflexive, symmetric, and transitive and therefore defines an equivalence relation on $X$. \\ 
For each $y\in Y$, form the set $[x]_y = \{x: f(x)=y\}$. If $x_1,x_2\in [x]_y$ then $f(x_1)=y=f(x_2)$, so $[x]_y$ are the equivalence classes of $\sim$. Observe that $[x]_y$ is nonempty for each $y$ since $f$ is surjective, and that $[x]_{y_1}\cap [x]_{y_2} =\emptyset$ if $y_1\neq y_2$ since equivalences classes form a partition of $X$. \\
From each $[x]_y$ select exactly one element and call this element $x_y$.\footnote{In the case where there are infinitely many $[x]_y$, I'm pretty sure this requires the axiom of choice.} Define a mapping $g:Y\to X$ such that if $y\in Y$ then $g(y)=x_y$. Since $[x]_y$ is nonempty, $g$ maps each $y\in Y$ to some element $x_y\in X$, and since $[x]_y$ are disjoint $g$ maps each element of $Y$ to exactly one element of $X$. Therefore, $g:Y\to X$ is a true mapping. \\
Let $y\in Y$. Then $g(y) = x_y$ and, since $x_y\in [x]_y$, $f(x_y)=y$ by the definition of $[x]_y$. So $f(x_y)=f(g(y))=y$. Therefore, since $y$ was arbitrary, $f \circ g$ $=$ id$_Y$. By Lemma 1, since $f$ and $g$ are mappings and $f \circ g$ $=$ id$_Y$, we also have that $g$ is injective. Thus, if $f:X\to Y$ is a surjective mapping then there exists an injective mapping $g:Y\to X$ such that $f \circ g$ $=$ id$_Y$.
\end{proof}

{\bf Lemma 3:} Let $f:X\to Y$ be bijective. Then there exists a unique bijective mapping $g:Y\to X$ such that $f\circ g$ $=$ id$_Y$ and $g\circ f$ $=$ id$_X$.
\begin{proof}[Proof of Lemma 3]
We can proceed as we did in Lemma 2 by defining precisely the same equivalence relation and the same equivalence classes, $[x]_y$. However, since $f$ is bijective, $[x]_y$ contains only one element for every $y\in Y$, so there is only one choice of $x_y$ to make. Therefore, we can define $g:Y\to X$ such that $g(y)=x$ if and only if $f(x)=y$. Since $f$ is surjective $g$ maps each element of $Y$ to an element in $X$, and since $f$ is injective $g$ maps each element of $Y$ to exactly one element of $X$. Therefore, $g:Y\to X$ is a true mapping. \\
Let $y\in Y$ such that $f(x)=y$ for some $x\in X$. Then we have $g(y)=x$ and $f(g(y))=f(x)=y$. Therefore, $f \circ g$ $=$ id$_Y$. \\
Now let $x\in X$ such that $g(y)=x$ for some $y\in Y$. Then we have $f(x)=y$ and $g(f(x))=g(y)=x$. Therefore, $g\circ f$ $=$ id$_X$. \\
Let $y_1,y_2\in Y$ and assume $g(y_1)=g(y_2)$. Then $y_1=f(g(y_1))=f(g(y_2))=y_2$. Therefore, $g$ is injective. \\
Let $x\in X$. Since $f$ is a mapping there is $y\in Y$ such that $f(x)=y$. This implies that $g(f(x))=g(y)$, and since $g\circ f$ $=$ id$_X$, $x=g(y)$. Therefore, $g$ is surjective. \\
We have shown that $g$ is both injective and surjective. Therefore, $g$ is bijective. \\
Clearly, $g$ is unique. If it were not then we could find another mapping $\tilde{g}:Y\to X$ such that $\tilde{g}(y)=x$ if and only if $f(x)=y$. But then we would also have $f \circ \tilde{g}$ $=$ id$_Y$ and $\tilde{g}\circ f$ $=$ id$_X$. So for each $y\in Y$, we would have $f(\tilde{g}(y))=y=f(g(y))$. In which case, $g(f(\tilde{g}(y)))=g(f(g(y)))$. But since $g\circ f$ $=$ id$_X$, we have that $\tilde{g}(y)=g(y)$ for every $y\in Y$. Therefore, $g:Y\to X$ is unique. \\
Thus, we have shown that if $f:X\to Y$ is a bijective mapping then there exists a unique bijective mapping $g:Y\to X$ such that $f\circ g$ $=$ id$_Y$ and $g\circ f$ $=$ id$_X$.
\end{proof}

{\bf Theorem:} If $f:X\to Y$ is bijective, then there exists a unique mapping $f^{-1}:Y\to X$ that is also bijective and satisfies:
\begin{enumerate}[(i)]
\item $f \circ f^{-1}$ $=$ id$_Y$
\item $f^{-1} \circ f$ $=$ id$_X$
\item $(f^{-1})^{-1}$ $=$ $f$
\end{enumerate}
\begin{proof}[Proof of Theorem]
By Lemma 3, if $f:X\to Y$ is a bijective mapping then there exists a unique bijective mapping $f^{-1}:Y\to X$, which we called $g$ in Lemma 3, such that
\begin{enumerate}[(i)]
\item $f \circ f^{-1}$ $=$ id$_Y$
\item $f^{-1} \circ f$ $=$ id$_X$
\end{enumerate}
We will call $f^{-1}$ \say{the inverse of $f$} and show that the inverse of $f^{-1}$,  denoted $(f^{-1})^{-1}$, is $f$. \\
By Lemma 3,  $f^{-1}$ is bijective if $f$ is bijective. Therefore, there exists a unique bijective mapping $g:X\to Y$ such that $f^{-1}\circ g=$id$_X$ and $g\circ f^{-1}=$id$_Y$. \\
Let $x\in X$. We have that $(f^{-1}\circ f)(x)=x$ since $f^{-1}$ is the inverse of $f$ and $(f^{-1}\circ g)(x)=x$ by Lemma 3. Therefore, $(f^{-1}\circ f)(x)=(f^{-1}\circ g)(x)$ and $f((f^{-1}\circ f)(x))=f((f^{-1}\circ g)(x))$. But $f\circ f^{-1}=$id$_Y$ so by the associativity of function composition, id$_Y(f(x))=$id$_Y(g(x))$ which implies $f(x)=g(x)$, for all $x\in X$. Therefore, $g=f$.
Thus, we have shown that if $f:X\to Y$ is a bijective mapping, the inverse of which is $f^{-1}$, then the inverse of $f^{-1}$ is precisely $f$. This proves part (iii) of the theorem.
\end{proof}
\end{document}














